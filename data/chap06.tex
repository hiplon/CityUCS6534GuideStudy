% !TeX root = ../FinalRepordCS.tex

\chapter{Conclusion and Future Work}

\section{Conclusion}
In conclusion, this study has delved into the realm of Internet of Things (IoT) with a focus on data security, particularly in industrial settings. 

The study has specifically explored methods and algorithms for physical layer key generation in wireless communication, with a spotlight on the LoRa protocol. The implementation of LoRa nodes, a RSSI-based physical key generation, a Chacha20 encryption in LoRa nodes, and various signal processing techniques has been presented, along with real-world tests in both outdoor and indoor environments. The results indicate the feasibility and effectiveness of the proposed system, providing a foundation for secure communication in IoT applications. Besides, I also get intersection of computer disciplines, project management\footnotemark,self-motivated improvement.

\footnotetext[1]{Project Management by Git: https://github.com/hiplon/CityUCS6534GuideStudy}

\section{Future Work}

While this study has laid the groundwork for LoRa Physical Layer Key Generation, there are avenues for future research and development:

Scalability and Interoperability: Investigate the scalability of the proposed system to handle a larger number of devices and its interoperability with diverse IoT ecosystems.

Optimization and Efficiency: Explore ways to optimize the key generation process for improved efficiency, considering factors such as computational resources and energy consumption.

Security Analysis: Conduct a thorough security analysis to identify potential vulnerabilities and propose countermeasures to enhance the robustness of the system.


Real-world Deployment: Conduct extensive real-world deployment in various industrial settings to validate the system's performance and reliability in diverse environments.

User-friendly Interfaces: Develop user-friendly interfaces and tools for easy configuration and management, making it accessible for a broader range of users.

Standardization Efforts: Contribute to standardization efforts in the field of IoT security, ensuring compatibility and adherence to industry standards.