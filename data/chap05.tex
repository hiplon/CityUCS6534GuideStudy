% !TeX root = ../FinalRepordCS.tex

\chapter{Implementation and Preliminary Performance Analysis}

\section{Implementation}
In this section, the primary emphasis is on executing the functionalities of LoRa nodes Alice and Bob by leveraging the Arduino programmable MCU board and IDE previously discussed. This implementation is conducted in tandem with the LoRa frequency modulation chip and the algorithms outlined earlier.

\begin{figure}
  \centering
  \subcaptionbox{LoRa Node Implementation\label{LoraNodeImplate}}
  {\includegraphics[width=0.8\linewidth]{LoraImplate.png}}
  \subcaptionbox{LoRa Transmission Implementation\label{LoraTransimissionImplate}}
  {\includegraphics[width=0.7\linewidth]{LORAtransmit.png}}
  \caption{Implementation Details}\label{Implementation Details}
\end{figure}

As depicted in Figure \ref{Implementation Details}, the LoRa nodes primarily executed the following functions:

\begin{itemize}
  \item Generation of RSSI key pairs 
  \item Generation of physical layer bit codes
  \item Implementation of Chacha20 encryption and decryption
\end{itemize}

Additionally, the transmission protocol for the physical layer key and encrypted communication channel has been realized, encompassing:

\begin{itemize}
  \item A half-duplex-based protocol for exchanging RSSI pairs  
  \item Chacha20-based transmission for the LoRa Protocol
\end{itemize}


\section{Deep-in-Building and Outdoor Test}
Following the implementation of the LoRa nodes, a practical test was conducted to generate LoRa physical layer keys within real-world application scenarios, encompassing both Deep-in-Building and Outdoor Tests. The test produced a wealth of results, providing tangible evidence for the practical applicability of LoRa in real-world settings. These findings hold significant reference value, serving to inform and guide the future deployment of the LoRa protocol, especially in the context of environmental monitoring.

As shown in Figure \ref{conghuatest}, in Liangkou Town, Conghua District, Guangzhou City, there is a sampling point (Bob) deployed on a communication tower for monitoring high-precision greenhouse gas flux. It is situated at an approximate height of 50 meters. Alice conducted an outdoor test from a temple located at a linear distance of around 300 meters from Bob.
Meanwhile, Figure \ref{panyutest} represents an outdoor test conducted at the central area of Guangzhou Higher Education Mega Center. This test involved a combination of stationary and mobile scenarios.
\begin{figure}
  \centering
  \subcaptionbox{Conghua out door test\label{conghuatest}}
  {\includegraphics[width=0.8\linewidth]{conghua.jpg}}
  \subcaptionbox{Panyu out door test\label{panyutest}}
  {\includegraphics[width=0.3\linewidth]{panyuhighereducationmegacenter.jpg}}
  \subcaptionbox{Near instruments test\label{neartheinstruments}}
  {\includegraphics[width=0.293\linewidth]{neartheinstruments.jpg}}
  \subcaptionbox{In building test\label{inbuildingtest}}
  {\includegraphics[width=0.3\linewidth]{inbuildingtest.jpg}}
  \caption{Deep-in-Building and Outdoor Test}\label{Test}
\end{figure}

Simultaneously, numerous tests were conducted indoors, especially in situations involving high-power instruments and electromagnetic field interference.
\subsection{RSSI Pairs Generation}
Through the aforementioned tests, our LoRa nodes utilized a half-duplex-based protocol for exchanging RSSI pairs. We collected a significant dataset, obtaining data in the form of \(n\) pairs \(\underbrace{\{RSSI_{Alice}^1, RSSI_{Bob}^1\}...\{RSSI_{Alice}^n, RSSI_{Bob}^n\}}_{n\_pairs}\) through the serial port or Arduino's storage card. Figure \ref{sampling5} shows the serial port displaying the sampling in LoRa Node.
\begin{figure}
  \centering
  \includegraphics[width=0.9\linewidth]{sampling5.png}
  \caption{Sampling in LoRa Node}
  \label{sampling5}
\end{figure}

However, due to the influence of various factors such as physical constraints, environmental conditions, communication distances, etc., further processing is required for the generated RSSI pairs.

\subsection{Key Generation}
Than a program that implemented constitutes a crucial section of the thesis, focusing on LoRa physical layer key generation. Beginning with the processing of RSSI data for both Alice and Bob, the code subsequently employs linear interpolation and Savitzky-Golay filtering to enhance the quality of the data.

Following these preprocessing steps, the script employs $M-ary$ quantization on the smoothed RSSI values, utilizing a specified number of bits per sample and an alpha parameter. The homotopy method for error correction is then implemented, where a matrix \(A\) and vector \(y\) undergo iterative updates to correct errors in the data.

The key generation process involves the random generation of a matrix \(A\), the computation of vectors \(y_1\) and \(y_2\) based on the quantized data from Alice and Bob, and the application of error correction through the homotopy method. The final step involves XORing the original bits with the corrected bits to derive the cryptographic key.

The code also incorporates visualizations using Matplotlib to illustrate the impact of linear interpolation and Savitzky-Golay filtering on the original RSSI values. Additionally, the thesis section concludes with the presentation of crucial output information, including filtered RSSI arrays for Alice and Bob, and the original and corrected keys for Alice.
\begin{figure}
  \centering
  \includegraphics[width=0.8\linewidth]{physicalkeygen.png}
  \caption{Result of Performing Key Generation}
  \label{physicalkeygen}
\end{figure}

In Figure \ref{physicalkeygen}, the key generation results incorporating multilevel quantization and error correction are illustrated. The figure demonstrates that both Alice and Bob generated a key through multilevel quantization. However, prior to error correction, the bit keys they generated were not identical. After Alice applied error correction, they successfully obtained the same key.

\subsection{Data Encrypt and Decrypt}
After the key has been generated, it can be utilized in conjunction with encryption algorithms to secure data for transmission. 
\begin{figure}
  \centering
  \includegraphics[width=0.3\linewidth]{ChaChaCipher.png}
  \caption{The ChaCha quarter-round function}
  \label{ChaChaCipher}
\end{figure}
This section focuses on the application of the previously mentioned Chacha20 algorithm, Figure \ref{ChaChaCipher}\cite{wiki:Chacha20}, to encrypt data transmission. As depicted in Figure \ref{Data Encrypt and Decrypt}, Alice sends emulated environmental monitoring data to Bob. Before transmitting the data, the Chacha20 algorithm is applied. Despite the broadcast nature of the data transmission, recipients without the key will only receive encrypted raw keystream of the data. Bob, possessing the same key as Alice, can perform the Chacha20 decryption on his device, thus obtaining the original message.

\begin{figure}
  \centering
  \subcaptionbox{Alice\label{Alice}}
  {\includegraphics[width=0.4\linewidth]{chachaalice.png}}
  \subcaptionbox{Bob\label{Bob}}
  {\includegraphics[width=0.7\linewidth]{chachabob.png}}
  \caption{Data Encrypt and Decrypt between Alice and Bob}\label{Data Encrypt and Decrypt}
\end{figure}

\section{Performance Analysis}
In this section, an analysis of sampling and performance tests, as well as an examination of password generation, were conducted based on references. The objective was to explore the impact of various parameters. Moreover, the results were visualized to enhance the understanding of the test data.

\subsection{Sampling Analysis}
I conducted an analysis of data derived from multiple RSSI sampling scenarios, which involved comparing data before and after interpolation, applying the SvGOL low-pass filter, and utilizing machine learning algorithms such as linear regression(LR) and Gaussian-distributed RBF regression. The Figure \ref{Sampling Data Analysis} illustrates that Alice and Bob can achieve consistent RSSI trends across various scenarios with an ample number of samples. However, it is noteworthy that they exhibit sensitivity to outliers in the data.

Furthermore, the application of the SvGOL filter to this signal evidently has an impact on mitigating the influence of outliers on the regression. These findings align with results from relevant cited studies and support the conjectures made in previous papers on this subject.
\begin{figure}
  \centering
  \subcaptionbox{LR - Conghua\label{conghuaanalystlr}}
  {\includegraphics[width=0.45\linewidth]{conghuaanalystlr.png}}
  \subcaptionbox{RBF - Conghua\label{conghuaanalystrbf}}
  {\includegraphics[width=0.45\linewidth]{conghuaanalystrbf.png}}
  \subcaptionbox{LR - Near Instruments\label{neartestanalystlr}}
  {\includegraphics[width=0.45\linewidth]{neartestanalystlr.png}}
  \subcaptionbox{RBF - Near Instruments\label{neartestanalystrbf}}
  {\includegraphics[width=0.45\linewidth]{neartestanalystrbf.png}}
  \caption{Sampling Data Analysis}\label{Sampling Data Analysis}
\end{figure}
The efficacy of an RSSI-based physical layer key generation algorithm is contingent on its ability to accurately capture trends and minimize the influence of outliers in the key generation process. In the context of the IoT domain, factors such as speed, accuracy, and ease of implementation also merit thorough consideration when evaluating the overall effectiveness of the algorithm.

\subsection{Key Generation Analysis}
The performance of physical layer key generation by using LoRa-Key under varying parameters are analyzed in this part.
\begin{figure}
  \centering
  \includegraphics[width=0.8\linewidth]{keygenanalyst.png}
  \caption{Key Generation Analysis}
  \label{keygenanalyst}
\end{figure}
As depicted in Figure \ref{keygenanalyst} and Formula \ref{alphaquan}, various \(\alpha\) values denote the guard band to data ratio, representing the excluded measurements in all the guard bands over the total measurements. A noticeable improvement in the bit agreement rate is observed after the implementation of the reconciliation technique, with the key agreement rate reaching 100$\%$ when \(\alpha\) rises to 1. 
\begin{equation} \int _{q_{i-1}}^{q_{i}-g_{i}} f_{a}da = \frac {1 - \alpha }{m}\quad \int _{q_{i}-g_{i}}^{q_{i}} f_{a}da = \frac {\alpha }{m - 1} {\ } where {\ } 1\leq i\leq {m−1}\label{alphaquan}\end{equation} 
The figures also reveal that the impact of reconciliation is consistent across different modes, underscoring the effectiveness and robustness of LoRa-Key in diverse environments.

\section{Summary}
Overall, the work presented in this implementation provides a solid foundation for the implementation and performance analysis of a RSSI-based physical key generation and LoRa-based encrypted communication system. The results and insights gained from real-world tests and analytical assessments contribute to the understanding of the system's capabilities and limitations, paving the way for further refinement and optimization.